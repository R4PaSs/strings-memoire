   \parskip=0pt
   \vspace*{0.1 truecm} 
   \begin{center}
    {\uppercase { REMERCIEMENTS }}\par
   \end{center}
   \nobreak \vspace*{1.10 truecm}
   \parskip=2ex

Je tiens à remercier tous les gens avec qui j'ai pu travailler ces dernières années, particulièrement Jean Privat, mon directeur de recherche, pour les cafés pris dans son bureau entrecoupés de discussions constructives et les enseignements qu'il m'a transmis.
Alexis et Alexandre, mes aînés au sein du Grésil, Romain (dont certains diront qu'il n'est plus stagiaire), Philippe, Fred (qui dit-on s'est transformé en plante), et l'ensemble des stagiaires qui sont venus travailler sur le projet Nit et leur patience face a mon syndrôme des jambes sans repos coordonnées a la musique (brutale, certains diront) que j'écoutais sans arrêt durant les trois années qu'ont duré mon travail.
Je remercie également mes amis, Mathieu, Alex, Geoffrey/Romain, les colocs qui ont contribué à ralentir mon travail à renfort de bières, ma foi délicieuses, et de journées oisives à regarder des films, mais sans qui la vie aurait été bien ennuyeuse; Florent, Marion, Nico, Anthony, Peggy, JP, Ariane, Arabelle, etc., qui n'ont pas manqué de me demander régulièrement comment mon mémoire avançait, qui ont toujours pris avec humour ma sempiternelle réponse "Ça recule pas!", et qui vont perdre leur éternel ami étudiant à la remise de ce papier.
Merci également à Louise Laforest, professeur et directrice du Département d'informatique ainsi qu'a son personnel, Kathleen, Mylène, Marie-Claude, pour les grands moments passés lors du maintenant traditionnel "Jeudi apporte ton lunch" et la quantité incroyable de sucreries apportées à chaque semaine; mon médecin n'est pas content, mais il n'y a toujours pas de traces de diabètes dans mon bilan sanguin il semblerait !
Merci {\fontspec{DejaVu Sans}�}galement {\fontspec{DejaVu Sans}�} UTF-8 pour sa coop{\fontspec{DejaVu Sans}�}ration {\fontspec{DejaVu Sans}�} l'{\fontspec{DejaVu Sans}�}criture de ce m{\fontspec{DejaVu Sans}�}moire.
Finalement, je remercie ma famille, et plus particulièrement mes parents, pour m'avoir encouragé à venir au Québec pour poursuivre mes études, et qui m'ont soutenu tout au long de ma vie.

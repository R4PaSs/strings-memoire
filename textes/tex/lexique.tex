\begin{lexique}
\texttt{AST}: Arbre de Syntaxe Abstrait, un arbre hétérogène organisant hiérarchiquement les entités gammaticales d’un texte

\texttt{BMP}: Plan Multilingue de Base, l'ensemble des 65536 premiers caractères d'Unicode, contenant les langues vivantes.

\texttt{Caractère}: Terme abstrait pouvant faire référence à un codet, un graphème ou un point de code.

\texttt{Caractère Combinant}: En terminologie Unicode, un caractère combinant est un point de code représentant un diacritique; il est combiné à d'autres caractères dits de base pour former un graphème.

\texttt{Caractères isomorphes}: Caractères ayant des glyphes équivalents, ex: Å (Angström) et Å (A-Ring).

\texttt{Chaîne plate}: Chaîne de caractères sous forme de tableau de caractères.

\texttt{Codet}: Plus petite entité d'un système de codage.

\texttt{Corde}: Chaîne de caractères sous forme d'arbre composé de noeuds de concaténation et de chaînes plates en feuille.

\texttt{Diacritique}: Signe accompagnant une lette ou un glyphe pour en modifier le son ou désambiguiser d'autres morts homonymes.

\texttt{Glyphe}: Représentation graphique d'un signe typographique, il s'agit de la mise en forme d'un graphème selon une police de caractères.

\texttt{Graphème}: Plus petite entité d'un système d'écriture.

\texttt{Ligature}: Fusion de deux graphèmes d’une écriture pour n’en former qu’un seul nouveau.

\texttt{Locale}: Ensemble de paramètres définissant l'environnement linguistique (langue, région, etc.) d'un utilisateur.

\texttt{Point de Code}: Valeur réservée d'un ensemble de caractères pour la représentation d'une entité textuelle.

\texttt{Unicode}: Standard informatique qui permet des échanges de textes dans différentes langues, à un niveau mondial. Il formalise les caractères des différents langages et les opérations effectuées sur ceux-ci, en incluant les problématiques de locales.

\texttt{UTF-8}: Unicode Transformation Format-8, forme de codage d'Unicode utilisant des codets de 8 bits.

\texttt{UTF-16}: Unicode Transformation Format-16, forme de codage d'Unicode utilisant des codets de 16 bits.

\texttt{UTF-32}: Unicode Transformation Format-32, forme de codage d'Unicode utilisant des codets de 32 bits.
\end{lexique}

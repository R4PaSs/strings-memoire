\begin{introduction}

La chaîne de caractères est un concept ancien, plus vieux que l'informatique elle-même.
Il s'agit de l'un des plus fondamentaux
de ce c\oe{}ur de métier, utilisé quotidiennement par des
millions de programmeurs.
Que ce soit pour des débutants s'essayant à leur premier \og Hello World \fg{}
qu'aux développeurs de compilateurs générant des milliers de lignes de code
en passant par des administrateurs tapant des commandes texte,
les chaînes de caractères sont partout.

Malgré leur importance, elles ont peu évolué depuis leur apparition.
Le plus souvent dans les langages impératifs et à objets, la représentation de
choix est un tableau de caractères.
La raison principale est l'efficacité de la solution.
En effet, les tableaux offrent une complexité temporelle constante pour l'accès
à un élément, les seules opérations à effectuer sont une addition pour calculer l'adresse
à charger et la lecture de la donnée.
Il existe cependant des différences fondamentales dans la façon dont les langages gèrent ledit
tableau: mutable ou immutable, codage utilisé en interne, micro-optimisations, etc.

Si cette approche est efficace pour l'accès indexé et l'itération, elle n'est pas exempte de défauts.
Nécéssitant de l'espace contigu pour fonctionner, elle dégénère lorsque
l'espace requis pour contenir les données de la chaîne atteint une grande taille.
Cette propriété rend des opérations lentes: la concaténation en est un exemple.
Une insertion à un endroit arbitraire de la chaîne impose de déplacer le contenu de la chaîne
pour permettre ladite insertion.
Selon la capacité disponible, cette opération peut entraîner une ré-allocation et copie du
contenu de la chaîne à un autre endroit de la mémoire, une opération effectuée en temps
linéaire.

Cette étude présente les chaînes de caractères dans le cadre de la bibliothèque
standard de Nit, un langage de programmation à objets.
Nous identifions les opérations fondamentales des chaînes de caractères, à partir
desquelles les autres opérations peuvent être dérivées.
Pour cette raison, ces opérations se doivent d'être aussi efficaces que possible.
Nous identifions également les forces et faiblesses des différentes structures
et proposons un moyen de les mitiger via un modèle original, que nous
avons implémenté en Nit.

\section{Contexte de l'étude}

Nit est un langage de programmation à objets, statiquement typé.
Il est une évolution de PRM, anciennement
développé au LIRMM, à Montpellier, France. Il s'agit d'un langage
destiné à l'expérimentation sur les langages à objets et les techniques
de compilation desdits langages.
Le langage, en tant que tel, est conçu pour concilier
l'expressivité et la rapidité de développement des langages de scripts avec
la performance et la sûreté des langages statiquement typés compilés.
Il supporte des fonctionnalités telles que les types nullables \cite{gelinas2009prevention},
la spécialisation multiple, les types virtuels, ainsi qu'une nouveauté
: le raffinement de classes \cite{privat2005raffinement}.
Le langage est également à vocation académique, le fait qu'il soit
nouveau et développé dans un cadre universitaire donne plus de
libertés pour tester et comparer de nouvelles approches.

Au début de cette étude, les chaînes de caractères étaient fournies dans
la bibliothèque standard de Nit sous forme de tableaux de caractères immutables.
Il existait également une forme mutable pour les cas de concaténations et
modifications efficaces.
Il n'existait pas de notion de codage et les chaînes étaient considérées comme
des blocs d'octets.

L'objectif de cette étude est de moderniser et d'améliorer la performance des chaînes
de caractères en gardant une API
\footnote{Application Programming Interface: Interface de programmation, ensemble de fonctions et services exposés aux utilisateurs d'un projet ou d'une bibliothèque}
aussi simple que possible.
Nous nous sommes focalisés majoritairement sur deux points: codage et structures de données.
Nous proposons une nouvelle API de manipulation des chaînes de
caractères au langage, basée sur une conjonction des
cordes et des chaînes plates.
Nous introduisons cette solution dans le chapitre \ref{solution_chap}.
L'ensemble des composants exposés de cette
API sont abstraits, afin de permettre le changement entre
les différentes représentations de façon automatique et transparente pour
l'utilisateur final.

\section{Problèmes étudiés}\label{probluxe8mes-uxe9tudiuxe9s}

\subsection{Comparaison des implémentations existantes des chaînes de
caractères}\label{intro_structs}

De l'ensemble des langages de programmation modernes, tous
proposent au moins une implémentation des chaînes de caractères.
Une majorité d'entre eux proposent une structure de données unique pour leurs
chaînes de caractères, d'autres proposent plusieurs façons de les
représenter.
Certains se basent sur des chaînes de caractères immutables,
d'autres en revanche proposent des chaînes mutables pour diverses raisons explorées
dans ce document.
En termes de codage également, des différences substantielles
existent dans la façon de gérer les chaînes en interne.

Dans ce document, nous comparons ces implémentations, tant au niveau de
l'utilisabilité que de la performance: identifier leurs forces,
leurs faiblesses, ainsi que les moyens proposés pour les mitiger.

\subsection{Chaînes de caractères et codages}

La capacité d'une bibliothèque de gestion des chaînes de caractères à traiter
du texte multilingue est un avantage important et passe par deux facettes:
codage et opérations.
Les différentes formes de codage proposent des solutions au problème
de la représentation en mémoire des chaînes de caractères dans des langues
différentes, chacun possédant ses forces et ses faiblesses.
Cependant, certains codages sont peu utilisés comme repré-sentation interne
du fait de leur inefficacité sur certaines opérations (accès indexé par exemple) dans
les chaînes de caractères, et ce, en dépit des avantages qu'ils proposent.
C'est le cas par exemple d'UTF-8, bien que ce dernier soit de plus en plus utilisé dans
les langages de programmation modernes.

Nit ne supportait pas de codage multilingue au début de ce travail.
Nous avons choisi UTF-8 comme représentation interne parmi
les principales solutions de codage disponibles aujourd'hui.
Ce document vise à justifier ce choix.
Nous prouvons également que la représentation en cordes peut
présenter une solution acceptable aux problèmes qu'un codage à longueur variable soulève.
Ces derniers sont détaillés dans la section \ref{unicode_troubles} de ce document.

\subsection{Performance et passage à l'échelle des chaînes de caractères}\label{intro_scaling}

Les chaînes de caractères à représentation
plate ont pour désavantage majeur d'être peu résistantes au passage
à l'échelle, et les traitements effectués sur elles peuvent dégénérer.
Le problème est plus important lorsque l'on parle de chaînes immutables,
chaque instance de chaîne possédant son propre tableau de caractères.
Les opérations de mutation (concaténation, modification, etc.) entraînent
une ré-allocation et une copie, très coûteuse lorsque sur une grande
chaîne, le coût de ces opérations augmentant linéairement.

Une structure alternative, les cordes peuvent limiter le problème soulevé ici en permettant, dans le cas
de modifications localisées sur des chaînes plates ou de concaténation,
de remplacer l'opération d'allocation et copie par la création d'un nouvel objet.
Cependant, la question se pose de leur impact sur les autres opérations,
notamment sur des chaînes courtes, à savoir est-ce que les cordes sont
toujours la bonne représentation pour ces cas là, ou bien la représentation
plate est-elle à préférer ?

Une solution pour limiter les problèmes des cordes et des chaînes plates
est de passer d'une représentation à l'autre passé une certaine taille.
Cette étude vise à déterminer à partir de quel seuil la
conversion vers des cordes devient bénéfique, et traite de la façon
dont la conversion s'effectue ainsi que des éventuels problèmes
pouvant provenir de ce comportement.

\subsection{Contributions}

Pendant cette étude, nous avons implémenté un nouveau modèle de représentation
des chaînes de caractères dans le langage Nit.
Cette contribution représente l'originalité principale résultant de cette étude.

Nous avons également ajouté le support d'Unicode au langage, en supportant les
manipulations sur des séquences codées en UTF-8.

Nous souhaitons attirer l'attention du lecteur sur le peu d'études scientifiques
disponibles concernant les implémentations des chaînes de caractères dans les
différents langages.
En effet, le champ de recherche au niveau académique est peu fourni en termes de
publications sur les implémentations dans les langages de programmation.
Une des contributions de cette étude est d'avoir étudié les documents spécificatifs des langages,
ainsi que le code source des implémentations disponibles pour étudier leurs forces et faiblesses.

Finalement, une grande partie du travail s'est focalisé sur des micro-optimisations
afin d'assurer un niveau de performances acceptable pour l'ensemble des manipulations
sur les chaînes de caractères en Nit.

\section{Organisation du document}

Le chapitre \ref{state_art_chap} détaille l'état de l'art des solutions existantes.
Nous y présentons les structures de données pour représenter les chaînes de caractères.
Nous étudions également les différents codages, et présentons leurs avantages et inconvénients.

Dans le chapitre~\ref{solution_chap}, nous poursuivons en présentant une solution aux problèmes de
cas dégénératifs que nous exposons en section~\ref{intro_scaling}.
Cette solution est entièrement implémentée en Nit, et est aujourd'hui utilisée
dans la bibliothèque standard du langage.
Ce chapitre fera également un état des lieux des contributions reliées à ce mémoire dans
la bibliothèque standard du langage Nit.

Le chapitre \ref{perf_chap}, mesure des performances, suivra la présentation de notre solution, des
contributions et problèmes rencontrés.
Il sert de validation en termes de performances, aussi bien dans des cadres de
micro-benchmarks, fait pour tester la robustesse des approches dans des cas dégénératifs,
que dans de vrais programmes Nit.

Finalement, nous concluons ce mémoire en résumant les avantages et inconvénients
de notre solution, et en mettant en perspective le travail restant à effectuer.

\end{introduction}

{\abstract

% en gros; contenu du document, originalité, valeur scientifique
% environ 300 mots

% contenu:
% but, nature et envergure de la recherche
% sujets traités
% hypothèses de travail et méthodes utilisées
% principaux résultats
% conclusion auquels ont est arrivé

% 4 ou 5 mots clés

Les chaînes de caractères sont des entités fondamentales des langages de programmation.
En représentant le texte, elles sont indispensables au travail des programmeurs, qu'ils soient en compilation, bases de données, interfaces homme-machine, traitement de texte, etc.
Il est primordial que ces structures soient aussi performantes que possible.
Malgré son importance, peu d'études se sont penchées sur ce pan de l'informatique et l'ensemble des techniques aujourd'hui utilisées ont peu évolué durant les trente dernières années.

Dans cette étude, nous étudions les implémentations actuelles des chaînes dans les langages de programmation,
tant au niveau des structures de données que des codages.
Nous développons un modèle objet de représentation des chaînes de caractères efficace et résistant au passage à l'échelle, resposant sur une combinaison de cordes et chaines plates, entièrement implémenté dans le langage Nit.
Cette combinaison nous permet de combler les problèmes connus des chaînes de caractères telles que représentées dans les langages passés et actuels.
Nous combinons cette approche à une facilité d'utilisation proche des chaînes classiques pour permettre aux développeurs de tous niveaux de bénéficier de ces avantages sans efforts particuliers.

Enfin, nous validerons notre approche par le biais de programmes de mesure de la performance, aussi bien sur des scénarios mettant en oeuvre les opérations fondamentales des chaînes de caractères, que dans de vrais programmes écrits en Nit.

Mots clés: chaînes de caractères, cordes, langages de programmation à objets
}

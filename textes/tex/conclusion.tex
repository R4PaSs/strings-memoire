\begin{conclusion}

Dans cette étude, nous traitons des implémentations des chaînes de caractères dans les langages de programmation et des problématiques de performance qui en découlent.
Nous avons identifié les cas dégénératifs d'utilisation des chaines plates:

\begin{enumerate}
	\item concaténation: la ré-allocation et la copie sur de grandes chaînes;
	\item accès indexé: dû aux codages à longueur variable, l'opération est en O(n);
	\item modification: pour les mêmes raisons que l'accès indexé, le déplacement jusqu'à la position à modifier, et
		la modification, qui peut entraîner le déplacement d'une partie de la chaîne.
\end{enumerate}

Pour chaque problème des chaînes plates, les cordes sont une réponse appropriée.
Cependant, elles aussi exposent des cas dégénératifs:

\begin{enumerate}
	\item sous-chaînage: le parcours et la création des nouveaux noeuds de la corde peuvent empêcher à l'opération de s'effectuer en un temps raisonnable;
	\item itération: le parcours de la corde ralentit l'opération.
\end{enumerate}

Nous avons décidé de mélanger les deux approches dans une solution originale afin de profiter des avantages des
deux structures, et de mitiger leurs inconvénients.
Nous avons implémenté cette solution dans le langage Nit, et la transition d'une structure à l'autre s'effectue
de façon transparente pour l'utilisateur lors d'une concaténation via un système de seuil de conversion.

La question du codage des caractères est également traitée lors de cette étude.
En 2016, tous les langages de programmation se doivent de supporter au moins un des codages prévus et spécifiés
par Unicode, que ce soit par défaut (Nit, Go, Python, Ruby, etc.) ou par le biais de bibliothèques (C, C++, etc.).
Ces choix ont cependant des conséquences en termes de performance ou d'utilisation mémoire.

Nous avons choisi en Nit d'utiliser UTF-8 par défaut, ce choix a été motivé par les facteurs de popularité
du codage et de compatiblitité avec les chaînes C.
Ces deux avantages nous garantissent de pouvoir continuer à utiliser des fonctions de manipulations de chaînes
de caractères implémentées en C pour des raisons de performance, mais également de limiter la probabilité
d'avoir un travail de conversion à effectuer lors de la lecture de texte.
UTF-8 est également le codage Unicode le plus économe en termes de mémoire\footnote{À l'exception du rare cas
de texte contenant uniquement des caractères asiatiques, où UTF-16 est plus économe.}, ce qui garantit
une consommation raisonnable.

Nous apportons un modèle de représentation original de chaînes de caractères, efficace et simple à utiliser.
Il garantit:
\begin{enumerate}
	\item La non-dégénérescence des opérations effectuées sur les chaînes de caractères
	\item Une meilleure gestion de la mémoire en limitant l'espace contigu nécessaire pour stocker une chaîne entière
	\item De meilleures performances de ramasse-miettes en évitant d'allouer des blocs de
		texte trop grands et en évitant de copier ou déplacer des données d'un endroit à l'autre de
		la mémoire lors de manipulations de chaînes de caractères.
\end{enumerate}

La solution que nous présentons est viable, comme le montrent les résultats que nous présentons dans le
chapitre~\ref{perf_chap} sur la performance, aussi bien sur de micro-benchmarks que sur des programmes réels.
En effet, cette solution limite les effets des cas dégénératifs dans tous les cas.
Cependant, son impact sur des programmes réels est variable.
Notons que si une accélération n'est pas toujours constatée, la performance reste égale ou tout du moins
proche par rapport à une approche utilisant uniquement des chaînes plates.
Elle accuse par exemple un ralentissement moyen de 8\% sur le cas d'analyse syntaxique JSON, un cas
défavorable à l'utilisation des cordes.

Dans le futur, la bibliothèque continuera à évoluer, de façon à améliorer ses performances
dans les cas présentés.
Nous continuons activement la recherche de meilleures heuristiques pour la production de cordes
dans des cas d'utilisation courants.

En termes de codage, il reste à explorer la possibilité de transiter vers une API orientée graphème,
à l'instar des implémentations de Swift et Perl6.
Cette modification aurait pour conséquence d'alourdir la représentation de caractère, n'étant plus
primitive.

Il est cependant possible d'imaginer une solution transformant \texttt{Char} en une structure abstraite
et en implémentant les différents cas: graphème simple (point de code), graphème étendu, etc.
À l'implémentation de cette solution, il faudra évaluer l'impact sur la performance, le passage
d'un type primitif à un type abstrait empêchant les optimisations à la génération du code (inlining
impossible, nécessité d'utiliser des appels polymorphes pour les fonctions de \texttt{Char}, etc.).

Le développement et l'optimisation de la version mutable de notre solution restent également à repenser
afin de réduire les problèmes inhérents aux codages à longueur variable dans les cas de modification et d'avoir
des performances proches des \texttt{buffers} actuels dans le cas de la concaténation.

Il reste encore également des possibilités d'évolution sur l'architecture elle-même.
Il est possible pour le sous-chainage par exemple, de proposer une implémentation adaptative, ré-allouant
et copiant les données sous un seuil, et partageant les sous-chaînes dans d'autres cas.

Une autre possibilité pour améliorer les performances est de se baser sur des techniques d'analyse de code
pour changer les représentations à la compilation ou ajouter des optimisations dans des cas limites.
\end{conclusion}
